\documentclass{article}
\usepackage{authblk}
\usepackage{graphicx}
\usepackage{amsmath, amssymb}
\usepackage{hyperref}

\title{A Data-Driven Exploration of E-Commerce Sales Patterns and Predictions}

\author[1]{Hasan Mhowwala}
\affil[1]{San Jose State University}

\begin{document}

\maketitle

\begin{abstract}
With the surge in e-commerce growth, understanding sales patterns becomes paramount. This paper offers an in-depth exploration of an e-commerce dataset, culminating in a predictive model for product sales. The study emphasizes the importance of a systematic data analysis and modeling approach, highlighting significant predictors for e-commerce sales.
\end{abstract}

\section{Introduction}
In the modern commercial landscape, e-commerce platforms are integral. Gleaning insights from sales data paves the way for effective marketing and inventory management. This research aims to identify patterns in e-commerce sales data and create an accurate sales prediction model.

\section{Data Understanding}
Our dataset encapsulates details of various products: their prices, stock levels, reviews, advertising budget, and corresponding sales. Initial observations revealed a distinction among three product categories: Electronics, Home Appliances, and Clothing.

\section{Data Visualization}
Visual explorations discerned a differential relationship between product categories and sales. Notably, Electronics and Home Appliances manifest a more extensive sales range compared to Clothing, suggesting potential variations in demand or marketing efficacy.

\section{Data Preparation}
Ensuring the data is well-suited for machine learning models required preprocessing:

\begin{itemize}
    \item \textbf{Encoding}: The 'Category' column underwent one-hot encoding, translating its categorical nature to a format amenable for algorithms.
    \item \textbf{Scaling}: Numerical attributes were standardized, targeting a mean of 0 and a variance of 1.
\end{itemize}

\section{Clustering Analysis}
The K-means algorithm was leveraged to cluster data points based on similarities. Using the silhouette score as a metric, the optimal cluster count was identified to be three. This segmentation provided a nuanced perspective of the dataset.

\section{Regression Modeling \& Evaluation}
Our primary objective was predicting sales. Three regressors were tested: Linear Regression, Decision Tree, and Random Forest. The Random Forest model outperformed its counterparts, as evidenced by the lowest MAE and RMSE values. Its ensemble-based methodology, which amalgamates predictions from various decision trees, rendered it most effective for this dataset.

\section{Results}
Scatter plots comparing actual versus predicted sales were generated:

\begin{itemize}
    \item The left plot (in blue) illustrates the default Random Forest model.
    \item The right plot (in green) elucidates the optimized Random Forest model.
\end{itemize}

These plots emphasized the proximity of predictions to actual sales, with the optimized Random Forest model displaying marginal improvements.

\section{Hyperparameter Tuning}
To further hone the model, hyperparameter tuning was conducted using GridSearchCV. This exercise yielded a marginally enhanced model.

\section{Conclusion}
This study accentuates the significance of structured data analysis in e-commerce. Key findings include:

\begin{itemize}
    \item Stock levels, product prices, and advertising budgets are cardinal predictors for sales.
    \item The Random Forest regressor is exceptionally suited for e-commerce sales predictions.
    \item Periodic model retraining with updated data is essential for maintaining predictive accuracy.
\end{itemize}

\section{Acknowledgments}
Gratitude is extended to all contributors and facilitators of this research.
\end{document}
