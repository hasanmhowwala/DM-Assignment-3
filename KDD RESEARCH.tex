\documentclass{article}
\usepackage{authblk}
\usepackage{graphicx}
\usepackage{amsmath, amssymb}
\usepackage{hyperref}

\title{Discovering Patterns in Car Rental Data: An Empirical Analysis}

\author[1]{Hasan Mhowwala}
\affil[1]{San Jose State University}

\begin{document}

\maketitle

\begin{abstract}
This research paper delves into the patterns hidden within a car rental dataset using the Knowledge Discovery in Databases (KDD) methodology. It illustrates the significance of data pre-processing, transformation, and mining techniques, highlighting how data-driven insights can be instrumental in predicting certain outcomes, such as car damages.
\end{abstract}

\section{Introduction}
The digitization of car rental processes has led to the accumulation of vast datasets, offering a myriad of opportunities for extracting actionable insights. This study focuses on a dataset comprising attributes such as pickup and dropoff locations, car type, payment methods, and reported damages upon car returns.

\section{Methodology}
The research employed the KDD methodology, comprising the following stages:
\begin{enumerate}
    \item Data Understanding \& Pre-processing
    \item Data Cleaning
    \item Data Transformation \& Feature Engineering
    \item Data Mining \& Model Evaluation
    \item Interpretation \& Recommendations
\end{enumerate}

\section{Data Understanding \& Pre-processing}
The dataset consisted of 650 entries spanning across 10 attributes. Data types encompassing numerical, categorical, and date-time attributes. Upon visual exploration, distributions of attributes like popular pickup and dropoff locations, car types, and payment methods were discerned.

\section{Data Cleaning}
Despite the comprehensive nature of the dataset, data quality assurance remains pivotal. Our findings are as follows:
\begin{itemize}
    \item Missing Values: The dataset demonstrated an absence of missing values.
    \item Outliers: Applying the Interquartile Range (IQR) method, anomalies were detected primarily within the 'Total\_Distance' and 'Total\_Amount' columns.
\end{itemize}

\section{Data Transformation \& Feature Engineering}
To make the dataset conducive to Machine Learning models, numerical attributes were standardized, categorical variables underwent one-hot encoding, and new attributes were introduced. Visual examinations showcased discernible rental patterns.

\section{Data Mining \& Model Evaluation}
The study revolved around predicting the propensity of damages being reported upon car return. Various classifiers were tested. Notably, the Decision Tree classifier was the most effective.

\section{Interpretation \& Recommendations}
Key insights derived from the model include rentals of longer durations or extensive distances showcasing a heightened risk of damages and the influence of car type and payment methods on damage reports.

\section{Conclusion}
The study underscores the implications of data analytics in sectors such as car rentals. The patterns unveiled can significantly aid in proactive decision-making processes.

\section{Acknowledgments}
The researcher extends gratitude to all contributors and datasets providers.

\end{document}

